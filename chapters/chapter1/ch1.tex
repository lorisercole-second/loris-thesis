\chapter{Introduction}

\begin{LEtext}
Struttura del capitolo di intro
\begin{itemize}
    \item Intro generale su trasporto termico, importanza nella tecnologia
    \item difficolta' nel descriverlo, fare modelli: 
    \item metodi per il calcolo della TC, Green-Kubo method
    \item ab initio GK - hurdles: gauge invariance, data analysis
    \item importanza per i vetri, il trasporto termico e' ancora poco capito nei vetri
    \item esempio: silica glass, importanza della TC, difficolta' coi potenziali classici
    \item scope of the thesis: tackle these theoretical and practical hurdles, and apply this method to amorphous materials for the first time, to the study of Silica glass
    \item lista della spesa
\end{itemize}
-----
\end{LEtext}
\bigskip

Heat flow is ubiquitous in nature, it governs a multitude of complex processes, from the evolution of stars and planets to the dynamical stability of biological systems, all the way down to the maintenance of optimal operating conditions in many (nano)technological applications. 
Heat flow determines the internal temperature distribution of out-of-equilibrium systems and the rate of cooling or heating of bodies, therefore the study of thermal transport is fundamental to modeling a multitude of complex systems, and to engineering nanotechnologies, where thermal insulation or thermal dissipation properties have to be properly designed. 
Yet, despite being one the oldest problems of statistical mechanics, a complete theoretical understanding of heat transport is still lacking. 

Transfer of thermal energy occurs via three different mechanisms, that may prevail of coexist in different regimes \cite{Lienhard2017}: \emph{convection}, in which heat is transported by a flow of mass; \emph{radiation}, in which heat is removed from the surface of the hot source by photons; and \emph{conduction}, in which heat transfer is determined by the microscopic dynamics of atoms, or in the case of metals, of conduction electrons. 
In condensed phases and at the molecular scale, conduction is by far the most relevant heat transfer mechanism, and we shall focus on it. 
The first macroscopic theory of heat transport was formulated by Fourier \cite{Fourier1878}, in 1822, who observed a proportionality law between the heat current $\mathbf{J}$ and the temperature gradient in the system $\nabla T$:
\begin{equation}
    \mathbf{J} = -\kappa\, \nabla T \,.  \label{eq:Fourier-law}
\end{equation}
The ratio between the heat flux and the temperature gradient defines the \emph{thermal conductivity}, $\kappa$, which is an intrinsic property of the material. 

Notwithstanding its fundamental importance, ``\emph{thermal conductivity has proven to be one of the most difficult transport coefficients to calculate}'' \cite{Evans1990} and its simulation is still today a conceptual, no less than practical, challenge to our materials modeling capabilities. 
In order to compute $\kappa$, one needs a microscopic theory that describes the conduction of heat carriers, \emph{i.e.} electrons and lattice vibrations (phonons). Hereafter we will focus only on the latter, the only ones contributing to heat transport in \emph{insulators} (the electrons following adiabatically in their ground state), to which we restrict our attention. 
The first microscopic theory of lattice thermal transport was formulated by Pierls, in 1929, and is based on the assumption that phonons obey the Boltzmann transport equation (BTE) \cite{Peierls1929}. 
About thirty years later, Green and Kubo (GK) expressed the thermal conductivity, as well as other transport coefficients, by liner response theory in terms of correlation functions of the heat currents \cite{Green1952,Green1954,Kubo1957a,Kubo1957b,Zwanzig1965}:
\begin{equation}
    \kappa \propto \int_{0}^{\infty}\!\langle{J}(t){J}(0)\rangle\, dt, \label{eq:GK-intro}
\end{equation}
where the brackets indicate ensamble averages, thus allowing its computation from simple equilibrium molecular dynamics (MD) simulations. 

In the meantime, a few decades ago, our abilities to understand and predict the properties of materials were boosted by the advent of density functional theory (DFT) \LE{[*ref*]}, which allows computing interactions entirely from quantum mechanics, thus freeing us from the need to leverage prior experimental knowledge, in order to perform MD simulations. 

Recent developments allowed the implementation of the BTE from first principles, thus making it the state of the art technique to compute $\kappa$ of bulk crystalline materials (\emph{e.g.} Si, Ge, diamond \cite{Broido:2007iu,Ward2009}) and nanostructures (\emph{e.g.} graphene and 2D materials \cite{Fugallo2014}). \emph{Ab initio} BTE also provides insights into the mechanisms of heat transport, by breaking down the contributions to $\kappa$ into single carrier properties. 
Nonetheless, the applicability of the BTE approach is limited to periodic materials at low temperatures, where the harmonic approximation of normal modes applies or anharmonic effects are very limited, and it cannot be straightforwardly applied to disordered systems, such as glasses and liquids, were phonons are not even defined, and for which MD is a natural choice. 

On the other hand, MD is set to overcome these limitations: it allows to study non-periodic and highly anharmonic systems in a straightforward way and to compute their thermal conductivity without making any assumption about the nature of thermal transport. The only inputs required are the atomic structure and an appropriate interatomic potential, which can be constructed empirically, \emph{e.g.} by fitting previous experimental or \abinitio results (classical MD), or directly by first principles DFT calculations (\abinitio MD, AIMD). 
Once one has these ingredients, $\kappa$ can be computed from equilibrium MD (EMD) or non-equilibrium MD (NEMD) simulations. The latter directly exploits the Fourier law, Eq.~\eqref{eq:Fourier-law}, and applies straightforwardly to finite systems and interfaces, but suffers from severe practical difficulties, such as finite-size and non-linear effects, that have to be carefully taken care of. 
Instead, we focus on EMD simulations, from which the thermal conductivity can be computed directly via the GK equation, Eq.~\eqref{eq:GK-intro}, that only requires an expression for the \emph{heat current}. For classical empirical potentials, such expression can be readily obtained from the atomic coordinates, velocities, and forces. On the contrary, only recently has an expression been successfully formulated in the framework of DFT for the first time \cite{Marcolongo2016}. 

Indeed, despite its rigour and simplicity, the GK theory has long been considered to be incompatible with accurate but expensive quantum simulation techniques like DFT, because the heat current entering Eq.~\eqref{eq:GK-intro} is ill-defined at the atomic scale, as the total energy cannot be decomposed into individual localized atomic contributions. 
This conceptual prejudice hindered the development of the GK theory in AIMD simulations for many years. It was only a few years ago that the spurious nature of this belief was recognized through the discovery of a \emph{gauge invariance principle} for transport coefficients \cite{Marcolongo2016,Ercole2016}. 
This principle ensures the value of thermal conductivity ultimately estimated through the GK equation does not depend on the microscopic details that define the the energy density, from which the heat current is derived, and hence is well-defined, as any measurable quantity should be. 
However, the problem of univocally defining the atomic energies exists also in classical MD simulations and was recognized in the past, although applications of the GK equation resorted on the most ``natural'' definition for them, without providing any formal justification. 
In the first part of this thesis we review this gauge invariance principle and extend it, proving its fundamental origin and its implications for the computation of $\kappa$.

On the other hand, despite these important discoveries, experience from classical simulations indicates that the practical implementation of the GK theory usually requires very long trajectories to estimate $\kappa$, thus making expensive quantum simulations unfeasible. 
The problem of efficiently estimating the GK thermal conductivity from finite-length trajectories is presented in the second part of this thesis. 
Several expedients have been used in the last twenty years to determine the convergence of the integral in Eq.~\eqref{eq:GK-intro} for different types of systems, yet it is very surprising that none of these is able to efficiently and reliably estimate $\kappa$ and its statistical error. Most of these methods are designed and tested on specific classes of systems, such as crystalline solids (for which, besides, the BTE is the preferable method), but do not work for liquids, disordered or strongly anharmonic systems, or they require extremely long MD simulations. 
We address this problem using an innovative data-analysis method based on the \emph{spectral analysis} of stationary time series \cite{Ercole2017}. Using this method it is possible to obtain an asymptotically unbiased and consistent estimator for $\kappa$ (\emph{i.e.} the bias and statistical error go to zero in the limit of long simulations), and we benchmark it numerically on four different classes of materials. We thus find that relatively short MD simulations, of the order of one or a few hundreds of picoseconds, are sufficient to obtain a very good statistical error on $\kappa$, of about $\sim 10\%$. This achievement finally demonstrates the feasibility of \abinitio GK simulations of heat transport, in particular for strongly anharmonic solids, liquids, and amorphous systems. 

\bigskip


\paragraph{Glasses}
\LEnote{* da finire: 
Thermal conductivity of glass systems is a fundamental property for many industrial and technological applications, \emph{e.g.} heat management in electronic devices, windows for green architectures.
However, ``\emph{thermal conductivity represents largely explored territories, ripe for new research efforts}'' \cite{MauroFM14,Mauro2014}. The understanding of thermal conductivity and its structural origin in glasses has been greatly overlooked in the literature. 
}

Vitreous silica has been the subject of significant research efforts in the last decades, due to is many technological applications that range from thermal insulation to laser engineering, semiconductor fabrication, and optical communication.
In particular, thanks to its excellent UV transparency, mechanical stability, and chemical durability, silica can be used in many optical applications, such as the diffractive elements and the protective windows of the optics assemblies of inertial confinement fusion facilities. In these facilities, extremely intense nanosecond laser pulses are used and can seriously challenge the durability the optical glasses. It is indeed well established that the small defects or impurities of the glass may cause local lattice heating and melting, resulting in damage craters that will rapidly degrade its optical performance \cite{Miller2004,Canaud2004,Miller2010,Chambonneau2014,Kuzuu1999,Stuart1995,Wong2006,Carr2010,Saito2000}. Moreover, these local damages can be mitigated by using pulsed laser treatments that increase the damage sites to temperatures of $2000-5000\un{K}$ in $10^{-9}$ to $10^{-12}\un{s}$ and partially restore the desired optical properties \cite{Soules2011}. 
The interpretation of these types of damage processes require the study of thermal properties and the prediction of the thermal conductivity of silica glass, especially in these extreme conditions that experiments cannot probe.

Furthermore, amorphous silica serves as the basis of multicomponent silica glasses, that are adopted for a wide range of special applications. 
Therefore, predicting the thermal conductivity of silica represent the first step towards the prediction of the thermal conductivity of more complicated glasses, whose components are characterized by a complex chemistry that more is difficult to model.

**Esempio vetri borosilicati ecc**

\LEnote{*** DA FINIRE con discorso sulla simulazione del trasporto termico nei vetri, difficolta' con potenziali classici, \abinitio, descrizione della nostra applicazione ***}

%\LEnote{**AEROGELS: \small Silica aerogel, a highly porous material first synthesized in the early thirties [1], are currently being produced using a sol–gel process such as hydrolyzing tetramethoxysilane (TMOS) to form silica and methanol, and subsequently dried through supercritical drying together with carbon dioxide [2]. Silica aerogel has several highly desirable properties including being environmentally safe, having high optical transmission as well as large thermal resistance [3]. These properties make silica aerogel very suited for applications such as thermal and acoustic insulation in buildings and appliances, passive solar energy collection devices, and dielectrics for integrated circuits [4]. Also, it is a suitable substitute for chlorofluorocarbon-based plastics in thermal insulation of refrigerators. The most well-known application was in Cherenkov radiators [5] as Cherenkov counters. Another crucial characteristic of aerogels is their extremely low density for a solid, which can go as low as 0.003 g/cm3. Comparatively, the density of air is approximately 0.0012 g/cm3 , which is only three times lower than that of the silica aerogel. This would represent significant weight savings when used in various monolithic structures. **}

%%%%%%%%%%%%%%%%%%%%%%%%%%%%%%%%%%%%%%%%%%%%%

%\section{Methods to compute the thermal conductivity}

\bigskip
\paragraph{Structure of this Thesis}
\LEnote{** Lista della spesa... ***}
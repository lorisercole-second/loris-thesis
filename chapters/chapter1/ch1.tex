\chapter{Introduction}

Struttura del capitolo  (vedi anche intros dei capitoli)
\begin{itemize}
    \item Intro generale su trasporto termico, importanza nella tecnologia
    \item difficoltà nel descriverlo, fare modelli: 
    \item metodi per il calcolo della TC, Green-Kubo method
    \item ab initio GK appena formulato, ma hurdles: Gauge invariance, data analysis
    \item Importanza per i vetri, il trasporto è ancora poco capito nei vetri
    \item esempio: silica glass, importanza della TC, difficoltà coi potenziali classici
    \item In this Thesis, we are going to tackle these theoretical and practical hurdles
    \item and apply this method to amorphous materials for the first time, study of Silica
\end{itemize}

%\paragraph{The Green-Kubo equation}
%\begin{equation}
%\kappa\propto\int_{0}^{\infty}\!\langle{J}(t){J}(0)\rangle\, dt, \label{eq:GK}
%\end{equation}

IDEE

GK perché no assumptionss about the nature of the thermal transport are required before predicting the thermal conductivity. The only inputs are an atomic structure and appropriate interatomic potnetial, which can be constructed from experimental and/or ab initio results. Atomic level observations are also possible. 
Challenges: finding suitable interatomic potentials, taking into accoun size effect

GK perche' anharmonic effects at high T


***********************************

\section{Glasses}
Thermal conductivity of glass systems is a fundamental property for many industrial and technological applications, \emph{e.g.} heat management in electronic devices, windows for green architectures.
However, ``thermal conductivity represents largely explored territories, ripe for new research efforts'' [J.Mauro...]\cite{MauroFM14,Mauro2014}. The understanding of thermal conductivity and its structural origin in glasses has been greatly overlooked in the literature. 

\paragraph{Vitreous silica}
Vitreous silica has been the subject of significant research efforts in the last decades, due to is many technological applications that range from thermal insulation to laser engineering, semiconductor fabrication, and optical communication.
In particular, thanks to its excellent UV transparency, mechanical stability, and chemical durability, silica can be used in many optical applications, such as the diffractive elements and the protective windows of the optics assemblies of inertial confinement fusion facilities. In these facilities, extremely intense nanosecond laser pulses are used and can seriously challenge the durability the optical glasses. It is indeed well established that the small defects or impurities of the glass may cause local lattice heating and melting, resulting in damage craters that will rapidly degrade its optical performance \cite{Miller2004,Canaud2004,Miller2010,Chambonneau2014,Kuzuu1999,Stuart1995,Wong2006,Carr2010,Saito2000}. Moreover, these local damages can be mitigated by using pulsed laser treatments that increase the damage sites to temperatures of $2000-5000\un{K}$ in $10^{-9}$ to $10^{-12}\un{s}$ and partially restore the desired optical properties \cite{Soules2011}. 
The interpretation of these types of damage processes require the study of thermal properties and the prediction of the thermal conductivity of silica glass, especially in these extreme conditions that experiments cannot probe.

Furthermore, amorphous silica serves as the basis of multicomponent silica glasses, that are adopted for a wide range of special applications. 
Therefore, predicting the thermal conductivity of silica represent the first step towards the prediction of the thermal conductivity of more complicated glasses, whose components are characterized by a complex chemistry that more is difficult to model.

**Esempio vetri borosilicati ecc**

\LEnote{**AEROGELS: \small Silica aerogel, a highly porous material first synthesized in the
early thirties [1], are currently being produced using a sol–gel process
such as hydrolyzing tetramethoxysilane (TMOS) to form silica and
methanol, and subsequently dried through supercritical drying together
with carbon dioxide [2]. Silica aerogel has several highly desirable
properties including being environmentally safe, having high optical
transmission as well as large thermal resistance [3]. These properties
make silica aerogel very suited for applications such as thermal and
acoustic insulation in buildings and appliances, passive solar energy
collection devices, and dielectrics for integrated circuits [4]. Also, it is a
suitable substitute for chlorofluorocarbon-based plastics in thermal
insulation of refrigerators. The most well-known application was in
Cherenkov radiators [5] as Cherenkov counters. Another crucial characteristic
of aerogels is their extremely low density for a solid, which can
go as low as 0.003 g/cm3
. Comparatively, the density of air is approximately
0.0012 g/cm3
, which is only three times lower than that of the
silica aerogel. This would represent significant weight savings when
used in various monolithic structures. **}

%%%%%%%%%%%%%%%%%%%%%%%%%%%%%%%%%%%%%%%%%%%%%

\section{Methods to compute the thermal conductivity}
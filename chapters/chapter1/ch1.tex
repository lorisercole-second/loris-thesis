\chapter{Introduction}

Struttura del capitolo  (vedi anche intros dei capitoli)
\begin{itemize}
    \item Intro generale su trasporto termico, importanza nella tecnologia
    \item difficoltà nel descriverlo, fare modelli: 
    \item metodi per il calcolo della TC, Green-Kubo method
    \item ab initio GK appena formulato, ma hurdles: Gauge invariance, data analysis
    \item Importanza per i vetri, il trasporto è ancora poco capito nei vetri
    \item esempio: silica glass, importanza della TC, difficoltà coi potenziali classici
    \item In this Thesis, we are going to tackle these theoretical and practical hurdles
    \item and apply this method to amorphous materials for the first time, study of Silica
\end{itemize}

Heat flow is ubiquitous in nature, it governs a multitude of \LE{dynamical} processes in living organisms and the evolution of star and planets, all the way down to the maintenance of optimal operating conditions in many nanotechnological applications. 
Heat flow determines the internal temperature distribution and the rate of cooling of heating of bodies, therefore the study of thermal transport is fundamental to understanding a multitude of complex systems, and to the engineering of technology, where thermal insulation or thermal dissipation properties have to be properly designed. 
Yet, despite being one the oldest problems of statistical mechanics, a complete theoretical understanding of heat transport is still lacking. 

Transfer of thermal energy occurs via three different mechanisms, that may prevail of coexist in different regimes: [8] \emph{convection}, in which heat is transported by a flow of mass; \emph{radiation}, in which heat is removed from the surface of the hot source by phtons; and \emph{conduction}, in which heat transfer is determined by the microscopic dynamics of atoms, or in the case of metals, of conduction electrons. 
In condensed phases and at the molecular scale, conduction is by far the most relevant heat-transfer mechanism, and we shall focus on it. 
The first macroscopic theory of heat transport was formulated by Fourier in 1822, who observed a proportionality law between the heat current $\mathbf{J}$ and the temperature gradient in the system $\nabla T$:
\begin{equation}
    \mathbf{J} = -\kappa\, \nabla T \,.  \label{eq:Fourier-law}
\end{equation}
The ratio between the heat flux and the temperature gradient defines the \emph{thermal conductivity} (TC), $\kappa$, which is an intrinsic property of the material. 

Notwithstanding its fundamental importance, ``\emph{thermal conductivity has proven to be one of the most difficult transport coefficients to calculate}'' \cite{Evans1990} and its simulations is still today a conceptual, no less than practical, challenge to our materials modeling capabilities. 
In order to compute $\kappa$, one needs a microscopic theory that describes the conduction of heat carriers, \emph{i.e.} electrons and lattice vibrations (phonons). Hereafter we will consider only the latter, which are the only ones contributing to the heat transport in \emph{insulators} (the electrons following adiabatically in their ground state), to which we restrict our attention in this work. 

The first microscopic theory of lattice thermal transport was formulated by Pierls in 1929, and is based on the assumption that phonons obey the Boltzmann transport equation (BTE) \cite{Peierls1929}. 
About thirty years later, Green and Kubo (GK) made use of liner response theory and expressed the thermal conductivity (as well as other transport coefficients) in terms of correlation functions of the heat currents \cite{Green1952,Green1954,Kubo1957a,Kubo1957b,Zwanzig1965}, thus allowing its computation from simple equilibrium molecular dynamics (MD) simulations. 

At the same time, some decades ago, our prediction capabilities were boosted by the advent of density functional theory (DFT), which allows computing interactions entirely from quantum mechanics, thus freeing MD simulations from the need to leverage prior experimental knowledge of these interactions. 

Recent developments allowed the implementation of the BTE from first principles, becoming the state of the art technique to compute $\kappa$ of bulk crystalline materials \LE{[broido, ward,..]} and nanostructures (\emph{e.g.} graphene and 2D materials \LE{[]Fugallo]}). \emph{Ab initio} BTE also provides insights into the mechanisms of heat transport, by braking down the contributions to $\kappa$ in single carrier properties. 
Nonetheless, the applicability of the BTE approach is limited to periodic materials at low temperatures, where the harmonic approximation of normal modes applies, or anharmonic effects are very limited, and it cannot be straightforwardly applied to disordered systems, such as glasses and liquids, were phonons are not even defined, and for which MD is a natural choice. 





%\paragraph{The Green-Kubo equation}
%\begin{equation}
%\kappa\propto\int_{0}^{\infty}\!\langle{J}(t){J}(0)\rangle\, dt, \label{eq:GK}
%\end{equation}

IDEE

GK perché no assumptions about the nature of the thermal transport are required before predicting the thermal conductivity. 
The only inputs are an atomic structure and appropriate interatomic potential, which can be constructed from experimental and/or \abinitio results. Atomic level observations are also possible. 
Challenges: finding suitable interatomic potentials, taking into account size effect

GK perche' anharmonic effects at high T


***********************************

\section{Glasses}
Thermal conductivity of glass systems is a fundamental property for many industrial and technological applications, \emph{e.g.} heat management in electronic devices, windows for green architectures.
However, ``thermal conductivity represents largely explored territories, ripe for new research efforts'' [J.Mauro...]\cite{MauroFM14,Mauro2014}. The understanding of thermal conductivity and its structural origin in glasses has been greatly overlooked in the literature. 

\paragraph{Vitreous silica}
Vitreous silica has been the subject of significant research efforts in the last decades, due to is many technological applications that range from thermal insulation to laser engineering, semiconductor fabrication, and optical communication.
In particular, thanks to its excellent UV transparency, mechanical stability, and chemical durability, silica can be used in many optical applications, such as the diffractive elements and the protective windows of the optics assemblies of inertial confinement fusion facilities. In these facilities, extremely intense nanosecond laser pulses are used and can seriously challenge the durability the optical glasses. It is indeed well established that the small defects or impurities of the glass may cause local lattice heating and melting, resulting in damage craters that will rapidly degrade its optical performance \cite{Miller2004,Canaud2004,Miller2010,Chambonneau2014,Kuzuu1999,Stuart1995,Wong2006,Carr2010,Saito2000}. Moreover, these local damages can be mitigated by using pulsed laser treatments that increase the damage sites to temperatures of $2000-5000\un{K}$ in $10^{-9}$ to $10^{-12}\un{s}$ and partially restore the desired optical properties \cite{Soules2011}. 
The interpretation of these types of damage processes require the study of thermal properties and the prediction of the thermal conductivity of silica glass, especially in these extreme conditions that experiments cannot probe.

Furthermore, amorphous silica serves as the basis of multicomponent silica glasses, that are adopted for a wide range of special applications. 
Therefore, predicting the thermal conductivity of silica represent the first step towards the prediction of the thermal conductivity of more complicated glasses, whose components are characterized by a complex chemistry that more is difficult to model.

**Esempio vetri borosilicati ecc**

\LEnote{**AEROGELS: \small Silica aerogel, a highly porous material first synthesized in the
early thirties [1], are currently being produced using a sol–gel process
such as hydrolyzing tetramethoxysilane (TMOS) to form silica and
methanol, and subsequently dried through supercritical drying together
with carbon dioxide [2]. Silica aerogel has several highly desirable
properties including being environmentally safe, having high optical
transmission as well as large thermal resistance [3]. These properties
make silica aerogel very suited for applications such as thermal and
acoustic insulation in buildings and appliances, passive solar energy
collection devices, and dielectrics for integrated circuits [4]. Also, it is a
suitable substitute for chlorofluorocarbon-based plastics in thermal
insulation of refrigerators. The most well-known application was in
Cherenkov radiators [5] as Cherenkov counters. Another crucial characteristic
of aerogels is their extremely low density for a solid, which can
go as low as 0.003 g/cm3
. Comparatively, the density of air is approximately
0.0012 g/cm3
, which is only three times lower than that of the
silica aerogel. This would represent significant weight savings when
used in various monolithic structures. **}

%%%%%%%%%%%%%%%%%%%%%%%%%%%%%%%%%%%%%%%%%%%%%

\section{Methods to compute the thermal conductivity}
\chapter{Conclusions}  \label{ch:conclusions}

\begin{itemize}
    \item GK for complex real systems, interfaces, ...
    \item neural networks per fittare le forze ma traiettorie \abinitio, cosi da riprodurre gli spettri e risolvere tutti i problemi per sempre
    \item improvement dei metodi di data-analysis:
    \item - improvements per analisi cepstrale -- diverse basi, per pesare di pi\`u le basse frequenze, ..., studio degli effetti dei vari filtri, uso del periodogramma modificato.
    \item - improvements gauge invariance -- definizione di correnti super scorrelate... a quel punto forse si trova qualche interpretazione pi\`u chiara
    \item Silica: raffinamento stime migliorando i metodi di analisi. Studio dipendenza conducibilit\`a da parametri strutturali (da definirsi adeguatamente, vedi Kob). Tests con force field nuovi (fittati sull'abinitio).
\end{itemize}
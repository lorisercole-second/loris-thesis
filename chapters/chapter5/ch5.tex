\chapter{Data analysis}  \label{ch:data-analysis}

\begin{LEtext}
The evaluation of transport coefficients in extended systems, such as thermal conductivity or shear viscosity, is known to require impractically long simulations. The Green-Kubo equation, Eq.~\eqref{eq:GK-complete}, expresses the thermal conductivity as the integral of the autocorrelation function of the heat flux, \emph{i.e.} an autocorrelation time. 

Many methods have been formulated to try to estimate its value from finite-length MD simulations: direct time-integration methods, fitting with exponential functions, and spectral methods; however, few of them provide a rigorous criteria to estimate the accuracy resulting from a given MD trajectory. Different classes of systems require different approaches to error analysis, but it is widely believed that all of them always require so long simulation times as to be unaffordable with accurate but expensive AIMD techniques \citep{Carbogno:2017gc}. 

The recent discoveries reinvigorated the interest in the quantum simulation of thermal conduction, and made it urgent to devise a data-analysis technique to make these simulations affordable, thus to paving the way to the \abinitio simulation of heat transport.
In order to solve this problem, we consider it in the light of the statistical theory of stationary time series, and we devise a data-analysis protocol leading to an asymptotically unbiased and consistent estimate of transport coefficients (\emph{i.e.} the bias and the statistical error can be made both arbitrarily small in the limit of long simulation times) and requiring shorter simulations than used so far. This protocol, based on the \emph{cepstral analysis} of time series, avoids any \emph{ad-hoc} fitting procedure and naturally provides an accurate estimate of the statistical error, thus lending itself to an easy implementation and automated use. 
While motivated by heat transport applications, our approach naturally applies to \emph{any} other transport properties that can be expressed, in a GK framework, in terms of time integrals of suitable autocorrelation functions, such as, \emph{e.g.}, ionic conductivities, viscosities, and tracer diffusivity, to name but a few. 

In this chapter we start by summarizing briefly the available techniques to estimate the Green-Kubo integral, then we formulate the problems in the light of the statistical theory of time series and show how to obtain an estimator of the thermal conductivity from ``cepstral analysis'', in the case of one-component and multi-component fluids. In the last section we validate this method with extensive benchmarks on the calculation of thermal conductivity.
\end{LEtext}


\section{Estimation of the Green-Kubo integral} \label{sec:data-analysis-methods}

In the following we shall focus on one-component systems, or molecular systems, where the heat flux correspond to the energy flux. 
In order to estimate the thermal conductivity from EMD with the GK approach, one starts by computing the energy current $\mathbf{J}$ from a MD trajectory in the microcanonical ensemble,
\footnote{\LEnote{***THERMOSTAT, INITIAL CONDITIONS STUFF***}} 
via the classical expression, Eq.~\eqref{eq:J-classical} (Eq.~\eqref{eq:J-classical-2body} for 2-body force fields), or the quantum DFT one, Eq.~\eqref{eq:DFT-Eflux}.

\begin{figure}
    \begin{center}
    ****************ESEMPI ARGON - ACQUA - SILICA - MGO ***********
        %\subfigure[\label{fig:argon-acf-long}]{\includegraphics[width=10cm]{chapters/chapter5/figures/handbook_argon_egauge_acf.pdf}}
        %\subfigure[\label{fig:argon-kappa-long}]{\includegraphics[width=10cm]{chapters/chapter5/figures/handbook_argon_egauge_kappa.pdf}}
    \end{center}
	\caption{(a) Time correlation function of the energy current of a liquid Argon, Eq.~\eqref{eq:JtJ0}
    (b) Integral of the time correlation function of liquid Argon, Eqs.~(\ref{eq:Lij-integralT}-\ref{eq:kappa-integralT}).
    The green and red lines are computed from a $100\un{ps}$ trajectory; the blue line is computed from a $1\un{ns}$ trajectory.
    Errors are estimated by standard block analysis.} \label{fig:acf-examples}
\end{figure}

The heat current autocorrelation function (HCACF), $\langle\mathbf{J}(t)\mathbf{J}(0)\rangle$, can then be estimated as a running average of the time-lagged current products, 
\begin{equation}
    \langle J^i(t) J^j(0)\rangle \sim \frac{1}{\mathcal{T}-t} \int_0^{\mathcal{T}-t} J^i(\tau+t) J^j(\tau) \, d\tau , \label{eq:JtJ0}
\end{equation}
where $\mathcal{T}$ is the length of the MD trajectory, and $J^i$ indicates any Cartesian component of $\mathbf{J}_i$.
Some typical examples of HCACFs are shown in Fig.~\ref{fig:argon-acf-long}, for four paradigmatic cases: liquid Ar, liquid H$_2$0, crystalline MgO and amorphous SiO$_2$. Liquid argon's HCACF is characterized by a simple exponential decay, as it is a simple diffusing liquid. The other systems, instead, are characterized by an initial drop, followed by a much longer one, with fast superimposed oscillations, due to the faster intramolecular vibrations, that can be linked to the optical phonon modes.

\begin{figure}
    \begin{center}
    ****************ESEMPI ARGON - ACQUA - SILICA - MGO ***********
        %\subfigure[\label{fig:argon-acf-long}]{\includegraphics[width=10cm]{chapters/chapter5/figures/handbook_argon_egauge_acf.pdf}}
        %\subfigure[\label{fig:argon-kappa-long}]{\includegraphics[width=10cm]{chapters/chapter5/figures/handbook_argon_egauge_kappa.pdf}}
    \end{center}
	\caption{(a) Time correlation function of the energy current of a liquid Argon, Eq.~\eqref{eq:JtJ0}
    (b) Integral of the time correlation function of liquid Argon, Eqs.~(\ref{eq:Lij-integralT}-\ref{eq:kappa-integralT}).
    The green and red lines are computed from a $100\un{ps}$ trajectory; the blue line is computed from a $1\un{ns}$ trajectory.
    Errors are estimated by standard block analysis.} \label{fig:kappa-examples}
\end{figure}


\subsection{Direct integration}
The evaluation of the GK integral, Eq.~\eqref{eq:GK}, or more generally of an Onsager coefficient $L^{ij}$, Eq.~\eqref{eq:L_def}, can then be performed by direct integration, as a function of the upper limit of integration: 
\begin{equation}
    L^{ij}(\mathcal{T}) = \frac{\rOmega}{k_B} \int_0^\mathcal{T} \langle J^i(\tau)J^j(0)\rangle \,d\tau .  \label{eq:Lij-integralT}
\end{equation}
One then recovers, via Eq.~\eqref{eq:multi_kappa}, an estimate for the thermal conductivity depending on $\mathcal{T}$:
\begin{equation}
    \kappa(\mathcal{T}) = \frac{1}{T^2} \frac{1}{(L^{-1}(\mathcal{T}))^{\smallone\smallone}}.  \label{eq:kappa-integralT}
\end{equation}
This function is usually very noisy: in fact, at times greater than the correlation time between $J^i$ and $J^j$, the correlation function $\langle J^i(\tau)J^j(0)\rangle$ approaches zero, hence $L^{ij}(\mathcal{T})$ starts integrating noise and behaves like the distance traveled by a random walk, whose variance grows linearly with the upper integration limit, as shown in Fig.~\ref{fig:kappa-examples}.
The evaluation of transport coefficients thus requires averaging over multiple trajectories (possibly multiple segments of a same long trajectory) and estimating the resulting uncertainty as a function of both the length of each trajectory and the upper limit of integration, usually with standard block analysis \cite{Frenkel2001}. This is a cumbersome task that often leads to a poor estimate of the statistical and systematic errors on the computed conductivity. All the more so when the signal is inherently oscillatory, due to the existence of high-frequency features in the power spectrum of the energy flux, possibly due to intramolecular oscillations that meddle with the noise and can make the convergence of the GK not obvious.

\begin{figure}
    \begin{center}
    ****************ACF KAPPA ARGON MCGAUGHEY pag.215***********
        %\subfigure[\label{fig:argon-acf-long}]{\includegraphics[width=10cm]{chapters/chapter5/figures/handbook_argon_egauge_acf.pdf}}
        %\subfigure[\label{fig:argon-kappa-long}]{\includegraphics[width=10cm]{chapters/chapter5/figures/handbook_argon_egauge_kappa.pdf}}
    \end{center}
	\caption{(a) heat current autocorrelation function, Eq.~\eqref{eq:JtJ0}, and
    (b) thermal conductivity as a function of the upper integration limit, Eqs.~(\ref{eq:Lij-integralT}-\ref{eq:kappa-integralT}), for three phases of LJ argon. Reproduced from Ref.~\cite{McGaughey2004a}.} \label{fig:argon-gk-phases-examples}
\end{figure}
\citeauthor{McGaughey2006} \cite{McGaughey2006,McGaughey2004a,McGaughey2004b} studied the case of LJ argon in different phases (fcc crystal, liquid, and amorphous). In Fig.~\ref{fig:argon-gk-phases-examples} we report the comparisons of their HCACFs and their direct GK integrals, Eq.~\eqref{eq:kappa-integralT}. At finite time the HCACFs show the two-stage decay \cite{Ladd1986}, and their extension decreases as the temperature increases, as one would expect due to decreasing phonons relaxation times. Their integral converge to the correspondent value of thermal conductivity, however the practical determination of $\kappa$ may be quite subjective. The ``first dip'' (FD) method, proposed by \citet{Li1998}, specifies its value by setting the upper limit of integration to be the first time where the HCACF goes negative. FD may give acceptable results in solid and liquid argon, but it is not suitable to the amorphous phase and in systems with multi-atom unit cell, as the HCACF oscillates wildly around zero before slowly decaying. Other methods to choose the cutoff time have been devised \cite{Chen2010}, but they are specifically conceived and tested for crystalline solids, such as silicon.


\subsection{Exponential fit and thermal conductivity decomposition}
Another technique proposed by some authors is the exponential fit (EF) method, in which a single or multi-exponential function is fitted to the HCACF beyond a certain point (determined on a case-by-case basis) \cite{Che2000a,Li1998,Zhang2015}. An example is the function \cite{Che2000a}:
\begin{equation}
    \langle J(t) J(0) \rangle = A_{ac,sh} \mathrm{e}^{-t/\tau_{ac,sh}} + A_{ac,lg} \mathrm{e}^{-t/\tau_{ac,lg}},
\end{equation}
where the subscripts $ac,sh$ and $lg$ refer to acoustic, short-, and long range. From here the thermal conductivity is then estimated as:
\begin{equation}
    \kappa = \frac{\rOmega}{k_B T^2} (A_{ac,sh}\tau_{ac,sh} + A_{ac,lg}\tau_{ac,lg})  =  \kappa_{ac,sh} + \kappa_{acf,lg}.
\end{equation}
\citet{McGaughey2004a} interpreted the two-stage behavior of solid argon's HCACF and the resulting decomposition of the thermal conductivity in the context of the mean phonon relaxation time: $\kappa_{ac,sh}$ corresponds to the phonons with lower relaxation times, whereas $\kappa_{ac,lg}$ corresponds to phonons with longer relaxation times, that make the longer decay time of the HCACF.
This model works pretty well \emph{e.g.} for solid argon \cite{McGaughey2004a}, diamond and carbon nanotubes \cite{Che2000a,Che2000b}, as well as in liquid argon, where a single-exponential decay is found, but does not work well to fit the long tails of the HCACF of silicon \cite{Schelling2002}.

The HCACF of amorphous argon, instead, shows a different behavior with respect to the solid an liquid phases. It is very similar to the velocity autocorrelation function and can be interpreted considering the different local environments the atoms experience. In a crystal, each atom is immersed in the same local environment; and the same is true in a liquid, if we average over time. Conversely, in an amorphous solid each atom has a different local environment: close its equilibrium position, it experiences the free trajectory of a liquid atom, at short time scales; but at slightly larger times it feels the interactions of the other atoms, that change its trajectory, and make the correlation negative. 
The first timescale of the HCACF decomposition, $\tau_{ac,sh}$, is related to the time it takes for the energy to move between nearest-neighbor atoms, and corresponds to the higher frequencies of the acoustic branches. The $\kappa_{ac,sh}$ is the only one important in the liquid and amorphous phase, and is a function of the coordination of the atoms \LEnote{**also found in silica structures \cite{McGaughey2004b}**} and scarcely depends on temperature, whereas $\kappa_{ac,lg}$ strongly depends on temperature.
In the amorphous case the EF method is not suitable for determining the thermal conductivity, and a direct integration should be performed instead. 
\LEnote{**CP limit?**}

In the case of unit cells with multiple atoms, where intramolecular vibrations manifest as fast oscillations in the HCACF, again, the EF method is not suited to determine $\kappa$. A rather arbitrary compromise is devised by \citet{McGaughey2004b}: they first filter the GK integral function, Eq.~\eqref{eq:Lij-integralT}, with a running average \cite{MovingAverage} and they choose the value at which it looks to converge. If the convergence is not clear they choose to stop the integration at the point at which the oscillations reach a minimum (neck). A more advanced fit can be performed in this case using the form \cite{McGaughey2004b}:
\begin{multline}
    \langle J(t) J(0) \rangle = A_{ac,sh} \mathrm{e}^{-t/\tau_{ac,sh}} + A_{ac,lg} \mathrm{e}^{-t/\tau_{ac,lg}} \\
        + \sum_i B_{op,i} \mathrm{e}^{-t/\tau_{op,i}} \cos(\omega_{op,i},t),
\end{multline}
\begin{equation}
    \begin{aligned}
        \kappa &= \frac{\rOmega}{k_B T^2} \left(A_{ac,sh}\tau_{ac,sh} + A_{ac,lg}\tau_{ac,lg} + \sum_i \frac{B_{op,i}\tau_{op,i}}{1+\tau_{op,i}^2\omega_{op,i}^2}\right) \\
        &= \kappa_{ac,sh} + \kappa_{acf,lg} + \kappa_{op},
    \end{aligned}
\end{equation}
where the subscript $op$ refers to optical phonon modes, however the procedure is not trivial. Furthermore, this procedure was found unsuitable for the case of amorphous silica, that therefore requires an alternative approach \cite{McGaughey2004b}.


\subsection{Spectral methods}
, or by extrapolating the power spectrum of the energy flux to the zero-frequency limit \citep{Volz2000}. 

\subsection{Other methods}
**discussione errori di sampling/integrazione, ... Mandadapu

\subsection{Final remarks}
We believe that no method is well suited
Gauge invariance --> ?

Others have attempted an error analysis of the MD estimate of the GK integral, based on either heuristic or rigorous arguments \citep{Jones2012,Wang_gk2017,Oliveira2017}, but they all require an estimate of an optimal value for the upper limit of integration, which determines a bias in the estimate, and which is in general difficult to obtain. Different classes of systems require different approaches to error analysis, but it is widely believed that all of them always require so long simulation times as to be unaffordable with accurate but expensive AIMD techniques \citep{Carbogno:2017gc}. In order to solve this problem, \cite{Ercole2017} considered it in the light of the statistical theory of stationary time series.



\section{Cepstral analysis}  \label{sec:cepstral-analysis}
%\section{Solids and one-component fluids} \label{sec:univariate}

In practice, MD gives access to a discrete sample of the flux process (a \emph{time series}), $J_n = J(n \epsilon)$, $0 \leq n \leq N-1$, where $\epsilon$ is the sampling period of the flux and $N$ the length of the time series, that we assume to be even. As was shown in Sec.~\ref{sec:Einstein}, the Wiener-Khintchine theorem allows one to express the heat conductivity in terms of the zero-frequency value of the power spectrum of the energy-flux (see Eqs.~(\ref{eq:Wiener-Khintchine}-\ref{eq:GK-S0})):
\begin{equation}
\kappa = \frac{\rOmega}{2k_B T^2} S (\omega=0). \label{eq:kappa-S0}
\end{equation}
Let us define the discrete Fourier transform of the flux time series as:
\begin{equation}
  \tilde{J}_{k}=\sum_{n=0}^{N-1} \mathrm{e}^{ 2\pi i\frac{kn}{N}} J_n, \label{eq:Jk}
\end{equation}
for $0 \leq k \leq N-1$.\footnote{Here, the convention for the sign in the exponential of the time-to-frequency Fourier transform is opposite to what adopted in Ref.~\cite{Ercole2017} and in most of the signal analysis literature, in order to comply with the convention for the space-time Fourier transforms usually adopted in the Physics literature and in Eqs.~\eqref{eq:kontinuity} and \eqref{eq:Fourier-continuity}.} The \emph{sample spectrum} $\hat S_k$, aka \emph{periodogram} in the statistics literature, is defined as
\begin{equation}
\hat{S}_{k}=\frac{\epsilon}{N} \left |\tilde{J}_{k} \right |^2, \label{eq:periodogram-def}
\end{equation}
and, for large $N$, it is an unbiased estimator of the power spectrum of the process, as defined in Eq.~\eqref{eq:Wiener-Khintchine}, evaluated at $\omega_k=2\pi\frac{k}{N\epsilon}$, namely: $\langle \hat S_k \rangle = S(\omega_k)$. The reality of the $\hat J$'s implies that $\tilde J_k=\tilde J^*_{N-k}$ and $\hat S_k=\hat S_{N-k}$, so that periodograms are usually reported for $0\leq k\leq \frac{N}{2}$ and their Fourier transforms evaluated as discrete cosine transforms.

The space autocorrelations of conserved currents are usually short-ranged. Therefore, in the thermodynamic limit the corresponding fluxes can be seen as sums of (almost) independent identically distributed stochastic variables, so that, according to the central-limit theorem, their equilibrium distribution is Gaussian. A slight generalization of this argument allows us to conclude that any conserved-flux process, like the heat flux, is Gaussian as well. The flux time series is in fact a multivariate stochastic variable that, in the thermodynamic limit, results from the sum of (almost) independent variables, thus tending to a multivariate normal deviate. This implies that at equilibrium the real and imaginary parts of the $\tilde J_k$'s defined in Eqs.~\eqref{eq:Jk} are zero-mean normal deviates that, in the large-$N$ limit, are uncorrelated among themselves and have variances proportional to the power spectrum evaluated at $\omega_k$. For $k=0$ or $k=\frac{N}{2}$, $\tilde J_k$ is real and $\sim \mathcal{N}\left (0, \frac{N}{\epsilon}S(\omega_k) \right )$; for $k\notin\left\{ 0,\frac{N}{2}\right\}$, $\mathfrak{Re}\tilde{J}_k$ and $\mathfrak{Im}\tilde{J}_k$ are independent and both  $\sim \mathcal{N}\left (0, \frac{N}{2 \epsilon}S(\omega_k) \right )$, where $\mathcal{N} (\mu,\sigma^2)$ indicates a normal deviate with mean $\mu$ and variance $\sigma^2$. We conclude that in the large-$N$ limit the sample spectrum of the heat-flux time series reads:
\begin{equation}
\hat{S}_{k} = S(\omega_k) \,\xi_{k} \,, \label{eq:periodogram-distribution}
\end{equation}
where the $ {\xi}$'s are independent random variables distributed as a $\chi_1^2$ variate for $k=0$ or $k=\frac{N}{2}$ and as one half a $\chi_2^2$ variate, otherwise. Here and in the following $\chi^2_\nu$ indicates the chi-square distribution with $\nu$ degrees of freedom. For the sake of simplicity, we make as though all the ${\xi}$'s were identically distributed as $\xi_k \sim \frac{1}{2} \chi_2^2$ for all values of $k$, thus making an error of order $\mathcal{O}(1/N)$, which vanishes in the long-time limit that is being assumed throughout this section.


\paragraph{Multiple samples}
In many cases of practical interest, multiple time series are available to estimate the power spectrum of a same process, $\{^{p\!}J_n\}$, $p=1, \cdots \ell$. For instance, in equilibrium MD a same trajectory delivers one independent time series per Cartesian component of the heat flux, all of which are obviously equivalent in isotropic systems. In these cases it is expedient to define a mean sample spectrum by averaging over the $\ell$ different realizations,
\begin{equation}
    \begin{aligned}
      {^{\ell\!}\hat{S}}_{k}& = \frac{\epsilon}{\ell N} \sum_{p=1}^{\ell}  \left |{^p\!}{\tilde J}_{k} \right |^2 \\
      & = S(\omega_k) {^{\ell\!}{\xi}_{k}} \,,
    \end{aligned}  \label{eq:mean-periodogram}
\end{equation}
where the ${^{\ell\!}\xi}$'s are $\chi_{2\ell}^2$ variates, divided by the number of degrees of freedom:
\begin{equation}
    ^{\ell\!}\xi_{k}\sim\frac{1}{2\ell}\chi_{2\ell}^{2}, \label{eq:chi-square-nu}
\end{equation}
for $k \notin \{ 0,\frac{N}{2} \}$.

Eqs.~\eqref{eq:periodogram-distribution} and \eqref{eq:mean-periodogram} show that ${^{\ell\!}}{\hat S_0}$ is an unbiased estimator of the zero-frequency value of the power spectrum, $\langle {^{\ell\!}}{\hat S_0} \rangle = S(0)$, and through Eq.~\eqref{eq:kappa-S0}, of the transport coefficients we are after.
However, this estimator is not consistent, \emph{i.e.} its variance does not vanish in the large-$N$ limit. This is so because a longer time series increases the number of discrete frequencies at which the power spectrum is sampled, rather than its accuracy at any one of them.

\begin{figure}
    \centering
    **** PERIODOGRAMS ***
    %\includegraphics[width=8cm]{chapters/chapter5/figures/handbook_water_psd.pdf}
    %\caption{Periodogram of a classical flexible model of water obtained from a $100\un{ps}$ MD trajectory. Grey: periodogram obtained directly from Eq.~\eqref{eq:mean-periodogram}, with $\ell=3$. Blue: periodogram filtered with a moving average window of width $1\un{THz}$, useful to reveal the main features of the spectrum (see text). The vertical dashed line delimits the low-frequency region used in the subsequent cepstral analysis.}  
    \label{fig:periodograms}
\end{figure}

Fig.~\ref{fig:periodograms} displays the periodogram of four paradigmatic cases (liquid Ar, liquid H$_2$O, crystalline MgO, and amorphous SiO$_2$) obtained from a $100\un{ps}$ ($200\un{ps}$ for MgO) classical MD trajectory, showing the extremely noisy behavior of the periodogram as an estimator of the spectrum. 
A consistent estimate of the value of the power spectrum at any frequency can be obtained by segmenting a time series into several blocks of equal length and then averaging over the sample spectra computed for each of them. When the length of the trajectory grows large, so does the number of blocks, thus making the variance of the average arbitrarily small. In practice, the determination of the optimal block size is a unwieldy process that leads to an inefficient determination of the length of the trajectory needed to achieve a given overall accuracy. 
Equivalently, a moving average \cite{MovingAverage} of the periodogram would consistently reduce the statistical noise, but the \emph{multiplicative} nature of the latter in Eq.~\eqref{eq:periodogram-distribution} makes it difficult to disentangle the noise from the signal and may introduce a bias. 
Here we adopt a different approach that allows us to obtain a consistent estimate of the zero-frequency value of the power spectrum from the statistical analysis of a \emph{single} trajectory sample (\emph{i.e.} no block analysis is needed) and such that the estimate of the trajectory length necessary to achieve a given accuracy is optimal.

\paragraph{Log-periodogram}
Spectral density estimation from finite empirical time series is the subject of a vast literature in the statistical sciences, embracing both parametric and non-parametric methods.\cite{Stoica2005} In the following we propose a semi-parametric method to estimate the power spectrum of a stochastic process, based on a Fourier representation of the logarithm of its power spectrum (the ``log-spectrum''). The advantage of dealing with the log-spectrum, instead of with the power spectrum itself, is twofold. First and foremost, the noise affecting the former is \emph{additive}, instead of multiplicative, thus making it simple and expedient to apply linear filters: limiting the number of components of the Fourier representation of the log-spectrum acts as a low-pass filter that systematically reduces the power of the noise and yields a consistent estimator of the log-spectrum at any given frequency. Second, as a bonus, the logarithm is usually smoother than its argument. Therefore, the Fourier representation of the logarithm of the power spectrum is more parsimonious than that of the spectrum itself.

Let $^{{\ell\!}}\hat{L}_{k} = \log(^{\ell\!}\hat{S}_{k})$ be the ``\emph{log-periodogram}'' of our time series. By taking the logarithm of Eq.~\eqref{eq:periodogram-distribution}, we can express $^{\ell\!}\hat{L}_{k}$ as:
\begin{equation}
  \begin{aligned}
    ^{{\ell\!}}\hat{L}_{k} &= \log (^{\ell\!}\hat{S}_{k} ) \\
    &= \log\left(S(\omega_k) \right) + \log( ^{\ell\!}{\xi}_k) \\
    &= \log\left(S(\omega_k) \right) + {^{\ell\!}\rLambda} + {^{\ell\!}{\lambda}}_{k},
  \end{aligned} \label{eq:log-PSD}
\end{equation}
where
\begin{equation}
    {^{\ell\!}\rLambda} = \left\langle \log( {^{\ell\!}{\xi}}) \right\rangle = \int_0^\infty \log\left (\frac{\xi}{2\ell}\right ) P_{\chi^2_{2\ell}}(\xi) \, d\xi = \psi(\ell)-\log(\ell) \label{eq:lambda-ell}
\end{equation}
is the expected value of the logarithm of the ${^\ell}\hat\xi$ stochastic variables defined in Eq. \eqref{eq:chi-square-nu}, $P_{\chi^2_{2\ell}}$ is the probability density of a $\chi^2_{2\ell}$ variate, $^{\ell\!}{\lambda}_k = \log\left( {^{\ell\!}{\xi}}_k\right) - {^\ell{\rLambda}}$ are zero-mean identically distributed independent stochastic variables, and $\psi(z)$ and is the digamma function \cite{PolyGamma}. 
The variance of the $^{\ell\!}\lambda$ variables is:
\begin{equation}
    \sigma_{\ell}^{2} = \int_0^\infty \log\left (\frac{\xi}{2\ell}\right )^2 P_{\chi^2_{2\ell}}(\xi) \, d\xi - \lambda_{\ell}^2 =\psi'(\ell),\label{eq:sigma2-ell}
\end{equation}
where $\psi'(z)$ is the tri-gamma function \cite{PolyGamma}. 

********DA QUI**********
Whenever the number of (inverse) Fourier components of the logarithm of the power spectrum is much smaller than the length of the time series, applying a low-pass filter to Eq.~\eqref{eq:log-PSD} would result in a reduction of the power of the noise, without affecting the signal. In order to exploit this idea, we define the ``\emph{cepstrum}'' of the time series as the inverse Fourier transform of its sample log-spectrum \citep{Childers1977}:
\begin{equation}
  ^{\ell\!} \hat C_{n} = \frac{1}{N}\sum_{k=0}^{N-1} {^{\ell\!} \hat L_{k}} \mathrm{e}^{-2\pi i\frac{kn}{N}}. \label{eq:sample-cepstrum}
\end{equation}
A generalized central-limit theorem for Fourier transforms of stationary time series ensures that, in the large-$N$ limit, these coefficients are a set of independent (almost) identically distributed zero-mean normal deviates \citep{Anderson1994,Peligrad2010}. It follows that:
\begin{equation}
  \begin{aligned}
    ^{\ell\!} \hat  C_{n} &= \lambda_{\ell} \delta_{n0} + C_{n} +  {^{{\ell\!}}{\mu}}_{n}, \\
    C_{n} &= \frac{1}{N}\sum_{k=0}^{N-1} \log\bigl (S(\omega_k) \bigr ) \mathrm{e}^{-2\pi i\frac{kn}{N}},
  \end{aligned} \label{eq:cepstrogram}
\end{equation}
where $^{{\ell\!}}{\mu}_{n}$ are independent zero-mean \emph{normal} deviates with variances $\left\langle {^{{\ell\!}}{\mu}_{n}^2}  \right\rangle$ $=\frac{1}{N}\sigma_\ell$ for $n\notin\left\{ 0,\frac{N}{2}\right\}$ and $\left\langle ^{{\ell\!}}{\mu}_{n}^{2}\right\rangle =\frac{2}{N}\sigma_{\ell}^{2}$
otherwise.
Fig.~\ref{fig:water-cepstrum} displays the cepstral coefficients of the low-frequency region of the spectrum of water (marked in Fig.~\ref{fig:water-periodogram}), showing that only the first few coefficients are substantially different from zero.

\begin{figure}
\centering
\includegraphics[width=8cm]{chapters/chapter5/figures/handbook_water_cepstrum.pdf}
\caption{Cepstral coefficients of water computed analyzing the low-frequency region of the periodogram (see Fig.~\ref{fig:water-periodogram}), defined in Eq.~\eqref{eq:sample-cepstrum}. }  \label{fig:water-cepstrum}
\end{figure}

Let us indicate by $P^*$ the smallest integer such that $C_n \approx 0$ for $P^* \le n \le N-P^*$. By limiting the Fourier transform of the sample cepstrum, Eq.~\eqref{eq:sample-cepstrum}, to $P^*$ coefficients, we obtain an efficient estimator of the zero-frequency component of the log-spectrum as:
\begin{equation}
  \begin{aligned}
    ^{{\ell\!}}\hat{L}_{0}^{*} & = {^{\ell\!}\hat{C}}_{0}+2\sum_{n=1}^{P^{*}-1}{^{{\ell\!}}\hat{C}}_{n} \\
    & = {^{\ell\!}\rLambda} + \log(S_0) + {^{{\ell\!}} {\mu}_{0}}+ 2 \sum_{n=1}^{P^*-1} {^{\ell\!} {\mu}_{n}}.
  \end{aligned} \label{eq:L0*}
\end{equation}
Inspection of Eq.~\eqref{eq:L0*} shows that $^{\ell\!}\hat{L}_{0}^{*}$ is a normal estimator whose expectation and variance are:
\begin{align}
	\langle {^{{\ell\!}}\hat{L}_{0}^{*}}\rangle &= \log(S_{0}) + {^{\ell\!}\rLambda}, \label{eq:L*} \\
	\sigma_\ell^{*}(P^{*},N)^{2} &=\sigma_{\ell}^{2}\frac{4P^{*}-2}{N}. \label{eq:sigma*}
\end{align}
Using Eq.~\eqref{eq:kappa-S0}, we see that the logarithm of the conductivity can be estimated from the cepstral coefficients of the flux time series through Eqs.~(\ref{eq:L0*}-\ref{eq:sigma*}), and that the resulting estimator is always normal with a variance that depends on the specifc system \emph{only} through the number of these coefficients, $P^*$. Notice that the absolute error on the logarithm of the conductivity directly and nicely yields the relative error on the conductivity itself.

The efficacy of this approach obviously depends on our ability to estimate the number of coefficients necessary to keep the bias introduced by the truncation to a value smaller than the statistical error, while maintaining the magnitude of the latter at a prescribed acceptable level. \cite{Ercole2017} proposed to estimate $P^*$ using the Akaike's information criterion (\cite{Akaike1974}), but other more advanced \emph{model selection} approaches \citep{Claeskens2008} may be more effective. This method consists in choosing $P^*$ as the one that minimizes the function:
\begin{equation}
\mathrm{AIC}(P)=\frac{N}{\sigma_\ell^{2}}\sum_{n=P}^\frac{N}{2} \hat{C}_{n}^{2}+2P. \label{eq:AIC-P}
\end{equation}
In Fig.~\ref{fig:water-filtered-psd} we report the low-frequency region of the spectrum of water obtained by limiting the number of cepstral coefficients to $P^*$:
\begin{equation}
^\ell\hat{S}_k^* = \exp\left[ 2\sum_{n=1}^{P^*-1} {}^\ell\hat{C}_n \mathrm{e}^{2\pi i \frac{k n}{N}} + {}^\ell\hat{C}_0 - {}^\ell\rLambda\right], \label{eq:filtered-psd}
\end{equation}
thus showing the filtering effect of this choice.
Finally, Fig.~\ref{fig:water-fkappa-Pstar} shows the value of thermal conductivity of water obtained through Eqs.~(\ref{eq:L0*}-\ref{eq:sigma*}).

\begin{figure}
    \centering
    \begin{center}
        \subfigure[\label{fig:water-filtered-psd}]{\includegraphics[width=8cm]{chapters/chapter5/figures/handbook_water_filtered_psd.pdf}}
        \subfigure[\label{fig:water-fkappa-Pstar}]{\includegraphics[width=8cm]{chapters/chapter5/figures/handbook_water_kappa_Pstar.pdf}}
    \end{center}
    \caption{
    (a) Filtered low-frequency region of the power spectrum of water obtained by limiting the number of cepstral coefficients to various values of $P^*$, Eq.~\eqref{eq:filtered-psd}. $P^*=7$ is the cutoff value suggested by the Akaike's information criterion, Eq.~\eqref{eq:AIC-P}. Grey: the unfiltered periodogram obtained from Eq.~\eqref{eq:periodogram-def}.
    (b) Thermal conductivity of water estimated from Eqs.~(\ref{eq:L0*}-\ref{eq:sigma*}) as a function of the cutoff, $P^*$. The colored bands indicate one standard deviation as estimated from theory. The vertical dashed line indicates the value suggested by the Akaike's information criterion, Eq.~\eqref{eq:AIC-P}.
    }
\end{figure}


\section{Multi-component fluids}
In Sec.~\ref{sec:multi-component} we have seen that in a fluid made of $Q$ atomic species there are in general $Q$ macroscopic fluxes interacting with each other through Onsager's phenomenological equations, Eq.~\eqref{eq:onsager}, not counting the different Cartesian components that do not interact amongst themselves because of space isotropy. A MD simulation thus samples $Q$ stochastic processes, one for each interacting flux, that we suppose to be stationary. These processes can be thought of as different components of a same multivariate process \citep{Bertossa2018}. As in Sec.~\ref{sec:univariate}, for the sake of generality we suppose to have $\ell$ independent samples of such a process, described by a multivariate time series of length $N$: $\{ ^{p\!}{J}^i_n \}$; $p=1,\dots \ell$; $i=1,\dots Q$; $n=0,\dots N-1$. Stationarity implies that $\langle {J}^i_n\rangle $ does not depend on $n$ and that $\langle {J}^i_n {J}^j_m \rangle$ only depends on $n-m$. We will further assume that $\langle {J}^i_n\rangle =0 $ and that $\langle {J}^i_n {J}^j_0 \rangle$ is an even function of $n$, which is the case when ${J}^i$ and ${J}^j$ have the same signature under time-reversal. By combining Eq.~\eqref{eq:multi_kappa} with Eq.~\eqref{eq:GK-S0}, we see that in order to evaluate the thermal conductivity in the multi-component case we need an efficient estimator for $\left ( S^{-1}_0\right )^{11}$, where $S^{kl}_0=S^{kl}(\omega=0)$ is the zero-frequency cross-spectrum of the relevant fluxes, ordered in  such a way that the energy one is the first.

Similarly to the one-component case, we define a mean sample cross-spectrum (or \emph{cross-periodogram}) as
\begin{equation}
 ^{(\ell Q)\!}\hat{S}_k^{ij} = \frac{1}{\ell} \sum_{p=1}^{\ell} \frac{\epsilon}{N} \left({}^{p\!}\tilde{J}_k^i\right)^* {}^{p\!}\tilde{J}_k^j .
\end{equation}
By discretizing Eq.~\eqref{eq:Sij(omega)} we see that $^{(\ell Q)\!}\hat{S}_k^{ij}$ is an unbiased estimator of the cross-spectrum, $\left \langle {}^{(\ell Q)\!}\hat{S}_k^{ij} \right \rangle = S^{ij}\left (\omega_k= \frac{2\pi k }{N\epsilon}\right )$. As it was the case for univariate processes, in the large-$N$ limit the real and imaginary parts of $\tilde J^i_k$ are normal deviates that are uncorrelated for $k\ne k'$. We conclude that the cross-periodogram is a random matrix distributed as a complex Wishart deviate \citep{Goodman1963a,Goodman1963b}:
\begin{equation}
  {}^{(\ell Q)\!}\hat{S}_k \sim \mathcal{CW}_Q \left(S(\omega_k), \ell\right). \label{eq:ComplexWishart}
\end{equation}
The notation $\mathcal{CW}_Q \left(S, \ell \right)$ in Eq.~\eqref{eq:ComplexWishart} indicates the distribution of the $Q\times Q$ Hermitian matrix
${}^{(\ell Q)\!}\hat{S}^{ij} = \frac{1}{\ell}\sum_{p=1}^\ell  {}^{p\!}{X}^i \, {}^{p\!}{X}^{j*}$,
where $\{ {}^{p\!}{X}^i \}$ ($p=1,\cdots\ell$, $i=1, \cdots Q$) are $\ell$ samples of an $Q$-dimensional zero-mean normal variate whose covariance is $S^{ij} = \langle X^i X^{j*} \rangle $.

Similarly to the real case, a Bartlett decomposition \citep{kshirsagar1959} holds for complex Wishart matrices \citep{Nagar2011}, reading:
\begin{equation}
{}^{(\ell Q)\!}\hat{S} = \frac{1}{\ell} \mathcal{S} R R^\top \mathcal{S}^{\dagger},  \label{eq:S_cholesky}
\end{equation}
where ``$\top$'' and ``$\dagger$'' indicate the transpose and the adjoint of a real and complex matrix, respectively; $\mathcal{S}$ is the Cholesky factor of the covariance matrix, $S= \mathcal{S} \mathcal{S}^{\dagger}$, and $R$ is a real lower triangular random matrix of the form
\begin{equation}
 R =
 \begin{pmatrix}
 c_1 & 0 & 0 & \cdots & 0\\
 n_{21} &  c_2 &0 & \cdots& 0 \\
 n_{31} &  n_{32} &  c_3 & \cdots & 0\\
\vdots & \vdots & \vdots &\ddots & \vdots \\
 n_{\smallQ1} & n_{\smallQ2} & n_{\smallQ3} &\cdots & c_\smallQ
 \end{pmatrix},
\end{equation}
where $c^2_i \sim \chi^2_{2(\ell-i+1)}$ and $ n_{ij}\sim \mathcal{N}(0,1)$. We stress that $R$ is independent of the specific covariance matrix, and only depends upon $\ell$ and $Q$. In particular it is independent of the ordering of the fluxes $J^i$. By expressing the $QQ$ matrix element of the inverse of $^{(\ell Q)}\hat{S}$ in Eq.~\eqref{eq:S_cholesky} as the ratio between the corresponding minor and the full determinant, and using some obvious properties of the determinants and of triangular matrices, we find that:
\begin{equation}
\frac{\ell}{\left({}^{(\ell Q)}\hat{S}_k^{-1}\right)^{\smallQ\smallQ}} = \frac{1}{\left(S_k^{-1}\right)^{\smallQ\smallQ}} c^2_\smallQ, \label{eq:S-1_choleskied}
\end{equation}
As the ordering of the fluxes is arbitrary, a similar relation holds for all the diagonal elements of the inverse of the cross-periodogram. We conclude that the generalization of Eq.~\eqref{eq:mean-periodogram} for the multi-component case is:
\begin{equation}
   ^{\ell}\hat{\underline{S}}_{\,k}\equiv\frac{\ell}{2(\ell-Q+1)}\frac{1}{\left( {}^{(\ell Q)}\hat{S}_k^{-1} \right)^{\smallone\smallone}} = \frac{1}{\left(S_k^{-1}\right)^{\smallone\smallone}} \, \xi_k, \label{eq:mean-multi-periodogram}
\end{equation}
where $\xi_k$ are independent random (with respect to $k$) random variables, distributed as
\begin{equation}
  \xi_k \sim
  \begin{cases}
    \frac{1}{\ell-Q+1} \,\chi^2_{\ell-Q+1}  \qquad & \mathrm{for} \; k \in \{0 , \frac{N}{2}\}, \\
 \\
 \frac{1}{2(\ell-Q+1)} \, \chi^2_{2(\ell-Q+1)} \qquad & \mathrm{otherwise}.
\end{cases}
\end{equation}
Starting from here we can apply the cepstral analysis as in the one-component case. The only difference is the number of degrees of freedom of the $\chi^2$ distribution, that becomes $2(\ell -Q+1)$, and a different factor in front of the result. Fig.~\ref{fig:grappa-periodogram} shows an example of multi-component power spectrum for a solution of water and ethanol.

\begin{figure}
\centering
\includegraphics[width=8cm]{chapters/chapter5/figures/psd_water_ethanol_40.pdf}
\caption{Multi-component power spectrum, as defined in Eq.~\eqref{eq:mean-multi-periodogram}, for a classical flexible model of a solution of water and ethanol $50\un{mol}\%$, obtained from a $100\un{ps}$ trajectory. Grey: $^{\ell}\hat{\underline{S}}_{\,k}$ obtained directly from Eq.~\eqref{eq:mean-multi-periodogram}, with $\ell=3$ and $Q=2$. Blue: $^{\ell}\hat{\underline{S}}_{\,k}$ filtered with a moving average window of width $1\un{THz}$ in order to reveal its main features. The vertical dashed line delimits the low-frequency region used in the subsequent cepstral analysis. Reproduced from \cite{Bertossa2018}.}  \label{fig:grappa-periodogram}
\end{figure}

\begin{figure}
    \begin{center}
        \subfigure[]{\includegraphics[width=8cm]{chapters/chapter5/figures/gk-grappa_capitolo_40.pdf}}
        \subfigure[]{\includegraphics[width=8cm]{chapters/chapter5/figures/convergence_water_ethanol_40.pdf}}
    \end{center}
	\caption{Convergence of the multi-component thermal conductivity estimator $\kappa$ using the direct time-integration approach and the cepstral method, for a classical flexible model of a solution of water and ethanol $50\un{mol}\%$, obtained from a $100\un{ps}$ trajectory.
(a) Direct time-integration approach in its Green-Kubo (green, as obtained from the matrix $L^{ij}(\mathcal{T})\propto\int_0^\mathcal{T} \left\langle J^i(t) J^j(0) \right\rangle dt $) and Einstein-Helfand (orange -- obtained from the  matrix $\left (L^{ij}\right )'(\mathcal{T}) \propto  \int_0^\mathcal{T}\left(1-\frac{t}{\mathcal{T}}\right) \left \langle J^i(t) J^j(0) \right \rangle dt$) formulations. The horizontal purple band indicates the value obtained by the cepstral method.
(b) Estimate of $\kappa$ with the cepstral method as a function of the number of cepstral coefficients, $P^*$, see Eqs.~(\ref{eq:L0*}-\ref{eq:sigma*}). The dashed vertical line indicates the value of $P^*$ selected by the AIC, Eq.~\eqref{eq:AIC-P}. Reproduced from \cite{Bertossa2018}.
}  \label{fig:twoCompConvergence}
\end{figure}

The method discussed so far shows a fundamental advantage with respect to a na\"ive implementation of direct time-integration approach.
Fig.~\ref{fig:twoCompConvergence} shows the two-component conductivity $\kappa$, obtained via Eq.~\eqref{eq:two-comp-kappa}, in the case of a water-ethanol solution, as a function of the upper time-integration limit $\mathcal{T}$ \citep{Bertossa2018}. Both the Green-Kubo and the Einstein-Helfand definitions of the finite-time expression of Onsager's coefficients (see Eq.~\eqref{eq:Einstein-Helfand}) are displayed.
Due to thermal fluctuations, the integral of the correlation function becomes a random walk as soon as the latter vanishes, eventually assuming any value. Therefore, there will be a set of times (see Fig.~\ref{fig:twoCompConvergence}) where the term $L^{\smallQ\smallQ}$ at the denominator in Eq.~\eqref{eq:two-comp-kappa} vanishes, leading to divergences in the evaluation of $\kappa$; an issue not affecting the one-component case. Hence, in such a formulation of the multi-component case, the mean value of the thermal conductivity estimator \textit{in the time domain} does not exist. On the contrary, the multi-component frequency-domain approach presented in this section, and built on sound statistical basis, provides a well defined expression for the estimator of $\kappa$ and its statistical error.


\section{Data analysis work-flow}
We summarize the steps leading to the estimation of thermal conductivity by the \textit{cepstral analysis} method, in order to highlight the simplicity of its practical implementation.
\begin{enumerate}
\item From a MD simulation compute the heat flux time series $J_n^1$ and the independent particle fluxes $J_n^q$, $q=2,\dots,Q$.
\item Compute the discrete Fourier transform of the fluxes, $\tilde{J}^{\small i}_k$, and the element $1/(\hat{S}^{-1})^{\smallone\smallone}$. In practice, only a selected low-frequency region shall be used (see \cite{Ercole2017} for a detailed discussion).\footnote{To lighten the notation, we drop the left superscripts of the variables in this subsection.}
\item Calculate $\log\left[1/(\hat{S}^{-1})^{\smallone\smallone}\right]$.
\item Compute the inverse discrete Fourier transform of the result to obtain the cepstral coefficients $\hat{C}_n$.
\item Apply the Akaike Information Criterion, Eq.~\eqref{eq:AIC-P}, to estimate the number of cepstral coefficients to retain, $P^*$.
\item Finally apply Eq.~\eqref{eq:L0*} to obtain $\hat{L}_0^*$, and evaluate the thermal conductivity as
\begin{equation}
\kappa = \frac{\rOmega}{2k_B T^2} \exp\left[\hat{L}_0^* - \psi(\ell - Q+1) + \log(\ell -Q+1) \right],
\end{equation}
and its statistical error as
\begin{equation}
\frac{\rDelta\kappa}{\kappa} = \sqrt{\psi'(\ell -Q+1) \frac{4P^{*}-2}{N}}.
\end{equation}
\end{enumerate}

\chapter*{Ringraziamenti}

Ricordo ancora, meno di quattro anni fa, il momento in cui iniziai ad avventurarmi in questo misterioso universo del trasporto termico. Mai avrei immaginato che in pochi anni questo progetto si sarebbe sviluppato e ramificato in così tanti e variegati aspetti, sia pratici, rivelatisi tutt'altro che banali e scontati, sia fondamentali, con inaspettati fondamenti teorici che stanno tutt'ora venendo a galla e che non smettono di affascinarmi. 
Senza dubbio, tutto questo percorso di ricerca non sarebbe stato possibile senza la guida e il sostegno del mio supervisor, Stefano Baroni, a cui va la mia profonda stima e gratitudine. È grazie a lui se mi sono appassionato a questo tema e alla ricerca. 

Devo un grosso ringraziamento anche ad Aris Marcolongo, con cui ho collaborato in molti aspetti di questo lavoro, e a Federico Grasselli e Riccardo Bertossa, con cui ho condiviso gli ultimi due anni del mio soggiorno alla SISSA, per le innumerevoli discussioni in macchina tra Trieste e Verona, e per l'aiuto e l'ispirazione in molteplici problemi.

Quattro anni a Trieste sono passati in un lampo, e questo lo devo interamente a tutti gli amici che hanno condiviso con me questo viaggio, chi prima o chi dopo, sono riconoscente a tutti voi: 
Mariami, Maja, Leyla, Sara, Federico, Riccardo, Ivan, Luca, Francesco, Simone, Tommaso, Caterina, Seher, Nina, Matteo, Juraj, Mattia, Stefano, Tommaso, Francesco, Nicolò, Lorenzo, Federico, Marco, e molti altri.

Nondimeno, devo ringraziare tutti gli amici di Verona, che pur vedendo sporadicamente, mi fanno sempre sentire a casa: i miei compagni Agorà, con cui abbiamo raggiunto traguardi incredibili (e spero sia solo l'inizio!), gli amici di mille più imprevedibili avventure, e gli amici storici su cui puoi sempre contare e che ho sempre il piacere di rincontrare. 

Infine, il ringraziamento più grande va alla mia famiglia e in particolare ai miei genitori, a cui dedico questo lavoro, e senza i quali non avrei davvero potuto raggiungere questo traguardo. \textbf{GRAZIE!}
